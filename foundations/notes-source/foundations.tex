\documentclass{article}

\usepackage{amsmath}
\usepackage{amssymb}
\usepackage{tcolorbox}
\usepackage{listings}

\newcommand{\prob}[2]{\textbf{Problem #1.} \textit{#2}}
\newcommand{\exec}[2]{\textbf{Exercise #1.} \textit{#2}}
\newcommand{\remark}{\textit{Remark.} }
\DeclareMathOperator{\charr}{char}
\newcommand{\qlim}[2]{\lim_{#1 \rightarrow #2}}

\lstset{language=Pascal} % has pretty much the same syntax highlighting as pseduocode

\title{Foundations}
\author{Michael Barz}

\begin{document}
\maketitle

Note: \textbf{$\log$ (or $\lg$) denotes the binary logarithm ($\log_2$), not $\log_{10}$ or $\log_e.$}

\section{Algorithms}

This is an interesting read, but nothing really noteworthy.

\section{Getting Started}

\subsection{Insertion Sort}

	Insertion sort will be our first algorithm, and it will solve the sorting problem:
\begin{tcolorbox}[title=Sorting Problem]
	\textbf{Input:} A sequence of $n$ numbers $(a_1, a_2, \dots, a_n).$

	\textbf{Output:} A permutation $(a_1', \dots, a_n')$ of the input sequence such that $a_1' \leq a_2' \leq \dots \leq a_n'.$
\end{tcolorbox}

\vspace{2mm}

The numbers sorted are called \textit{keys}.

Insertion sort

\begin{lstlisting}[frame=single]
for j=2 to A.length
	key = A[j]
	// Insert A[j] into the sorted sequence A[1..j-1]

\end{lstlisting}

\end{document}
